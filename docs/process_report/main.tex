\documentclass{article}
\renewcommand{\familydefault}{\sfdefault}
\usepackage{helvet}
\usepackage[utf8]{inputenc}

\usepackage{graphicx}
\graphicspath{{images/}}

\usepackage{caption}
\usepackage{subcaption}
\PassOptionsToPackage{hyphens}{url}\usepackage{hyperref}
\def\UrlOrds{\do\*\do\-\do\~\do\'\do\"\do\-}%
\usepackage{float}
\usepackage{gensymb}
\usepackage{amsmath}
\graphicspath{{images/}}
\usepackage[margin=1.0in]{geometry}

\begin{document}
\begin{titlepage}

\newcommand{\HRule}{\rule{\linewidth}{0.5mm}} % Defines a new command for the horizontal lines, change thickness here

\center % Center everything on the page
\textsc{\LARGE University of Edinburgh}\\[1.5cm] % Name of your university/college
\textsc{\Large System Design Project}\\[0.5cm] % Major heading such as course name
\date{January 2017}
\textsc{\large }\\[0.5cm] % Minor heading such as course title
\HRule \\[0.4cm]
{ \huge \bfseries Progress Report}\\[0.4cm] % Title of your document
\HRule \\[1.5cm]
\begin{minipage}{0.7\textwidth}
\begin{flushleft} \large
\emph{Group 10 Members:}\\
s1368635 \textsc{Deirdre Bringas}
\\s1452672 \textsc{Demetra Charalambous}
\\s1411707 \textsc{Joshua Green}
\\s1342226 \textsc{Nicholas Georgiou}
\\s1431686 \textsc{Tom Lutzeyer}
\\s1421057 \textsc{Titas Skrebe}
\\s1441731 \textsc{Vlad Buzatu}
\end{flushleft}
\end{minipage}\\[4cm]
\textsc{\Large NIGEL'S PHOTO}\\[0.5cm] 

\end{titlepage}

%-------------Introduction Section--------------------------------------------%
\newpage


\section{Introduction}
As part of the course "System Design Project" we are tasked to build and design
the code for an autonomous robot by using Lego, an Arduino and our programming
skills in order to play two-a-side football. Loosely following rules based on
the Robocup competition. We are assigned the group number 10; the members of
which are listed on the cover. After the third friendly match we will also have
to cooperate with group number 9, as they will be our teammates in future
matches forming "Team E". At the start of the semester we were assigned several
individual, group and team goals. These sub-goals would help us achieve our main
goal, which is having a fully-functional robot that can play football according
to the rules of the course. This report aims to cover how, as a group, we set
out to tackle this project. It also contains information on how we chose to
communicate, how tasks are allocated, when meetings are carried out, and how our
progress is tracked. Also included is information about how we allocate and use
resources, such as our budget as well as a risk assessment of problems that we
are, or possibly will, face in the future.

\section{Organisation}
\subsection{Communication}
A variety of ideas were put forward as our chosen communication method. Among
others Facebook messenger, email, irc, and Slack were the most popular. Soon
after we decided that Slack was the most appropriate method of communication.
Slack allows for uploading files and is the platform used for general
communication amongst the SDP students and mentors which were the deciding
factors in our choice. Also a Google calendar was created where all the group,
"Team E", mentor meetings and milestones will are added. This allows all members
of the group to keep track of dates and times of relevant events to attend in a
centralised manner. The Project Manager (for whom more details can be found
below) is in direct contact with our mentor, as well as our group members. The
Project Manager's primary responsibilities is distributing workloads, arranging
the group meetings, and verifying that everyone is meeting their deadlines.

\subsection{Task Allocation}
Task allocations were carried out once we decided on our goals and milestones
for this project. These goals included; having a working design for our robot by
the first friendly match (\textit{2017-2-1}), being able to catch the ball, turn
towards goal and shoot, and having a working vision system. With these clear
goals in mind, we allocated each member of the group a task, both individual and
group tasks. We decided that having a project manager would be essential for
group coordination and organisation. The role was assigned to Deirdre. With a
project manager in place further allocations were carried out. Titas and Joshua
chose to work on the hardware portion of the project, with Demetra, Nicholas,
Tom, Deirdre and Vlad working on the software part of the project. Once everyone
was allocated to a section of the project, the next step was to decide what each
person would work on and what deadlines would be set for delivering the product.
This was done by looking at what had to be done for each milestone and producing
a temporary table in which everyone was assigned a task and a date by which this
would be completed. This table is a rough representation of how tasks will be
completed as changes in the implementation of the robot might appear because of
unforeseen issues. Nonetheless, we found it a useful and efficient way of making
sure all members have a target to work towards and way to evaluate that. In
Figure 1, a single group member's draft table is shown, roughly displaying the
tasks until the second friendly match.

\begin{figure}[H]
	\centering
	\begin{minipage}{1\textwidth}
		\centering
		\includegraphics[width=16cm, height=6cm]{task_D.png}\\
		\caption{Example of individual temporary task table}
	\end{minipage}%
\end{figure}

We have also constructed a chart displaying how the members of the group are
allocated in each sub-group. Sub-groups are smaller groups of group 10 which
their members have tasks in common e.g. hardware group. After each friendly
match we have a group meeting where we talk about what is required to do next
and each member's tasks. During these meetings, responsibilities are assigned
and the group is each time divided into sub-groups. In the first meeting we
decided that an equal number of members in each sub-group would be ideal, as
both hardware and software were of major importance. We also planned ahead and
started the process report with two members of the group working on it (Figure 1
- part A). The first friendly match demonstrated that working on our overall
strategy was key for the next match, thus more members were focusing on the
software while still having members in the hardware group (Figure 1 - part B).
After the second match, our main aim is to establish a stable cooperation with
group 9 and to implement the changes needed on both software and hardware, in
order to achieve team-playing as group E. As the number of changes in strategy
would be more than the changes in the hardware, most members of the group are
working on the software. (Figure 1 - part C).

\begin{figure}[H]
	\centering
	\begin{minipage}{1\textwidth}
		\centering
		\includegraphics[width=16cm, height=17cm]{task_allocation.png}\\
		\caption{Diagram displaying how each member of the group is part of smaller group indicating the tasks to be done before each friendly match. Circles indicate the sub-groups that group 10 forms.}
	\end{minipage}%
\end{figure}


\subsection{Meetings}
Meetings are a vital component to the organisational structure of a group-based
project, as this is where we can all discuss the progress we've made and the
issues that have come up. Once a week, we meet with our mentor to relay how the
group is progressing as a whole, as well as receive advice regarding the
project. Additionally, sub-groups meet throughout the week to work on their
assigned tasks. The progress made is noted in our Slack channel. Once we have
collaborated with the group 9 in our group, we will also organise group meetings
where both groups will come together and discuss ongoing changes and how to
implement our tactics for match day.

\subsection{Tracking Progress}
Tracking progress is done through meetings as discussed above but also through
our slack channel and Git repository. Each group member should have a task
assigned to them which they should have completed by X date. Up to the set
deadline each member will updated the group on their work and whether or not
they are dealing with problems which may delay the deadline. When such issues
arise a group member who may have completed, or is close to completing their
task can help the other group member to solve the problem. When facing problems,
we are all in the group together to help troubleshooting and keep our project on
track.

\subsection{Git Repository}
Working as a group means many changes will be made on the code as each
individual makes changes to code. This way everyone can keep track of what
changes are being made. This also helps in tracking progress made towards our
next milestone.


\section{Individual milestones and group tasks}
Each individual in the group will set themselves personal or group targets if
they are working in a group of two for milestones set by them. For example if
two people are working on the strategy of the robot, then the first milestone
would be to have a basic strategy for the first Friendly game. As a group our
first Milestone is having a working robot by Monday 30th January which will be
able to perform in the first friendly. This will give us time to make
adjustments to problems we encounter on the robot before the first friendly.

As a group we have also set ourselves 'group Milestones' where we have a target
day to have completed a certain functionality of the robot. This will give
everyone the encouragement needed to push for their personal milestones so as
when the group Milestone is reached all members of the group have their
contribution towards this.

We have decided to have practice games with the group 9 in our group as this
will benefit both groups for testing the robots. It will also help us decide on
what strategies we will implement when we have to work as a group on match day
as through these practice games we will have a good understanding of each
group's strengths and weaknesses. For example, if our group is better at
defending than attacking or vice-versa then this will influence which robot will
defend or attack on match day.

Individual and group milestones can be found in the figures 3 and 4 in the
section "Gantt Charts", were are displayed using the symbol "*".

\section{Gantt Charts}
For better organisation of the group we decided to use Gantt charts. For this
purpose we are using "Smartsheet" on-line tool. These charts will help us to
complete the individual milestones faster and always be aware of what tasks are
to be done and until when. Although we are aware that the plan is possibly going
to change through the process, we created these charts to help us organising the
work. As we are approaching the first and the second friendly match, we created
two detailed Gantt charts, as we already know what needs to be done. For the
weeks coming after these matches, we created more general Gantt charts because
we cannot predict in detail what the workload will be.


\subsection{Tasks until first friendly match}
In this detailed Gantt chart illustrated below, the tasks and milestones that
need to be completed before the first friendly match are shown. In the first
week, we had a general meeting on Wednesday where we met with our group members.
We then separated into software and hardware groups and attended the
corresponding workshops. After the workshops tasks for each individual in the
group where assigned and each one started working on their allocated tasks.


The hardware group was responsible for building the basic robot structure, the
kicker and catcher of the robot and then testing each part of the hardware.
Meanwhile, the software group focused on the code from previous years so that
they could modify it later. After the basic structure has finished, the software
group cloned "Fred's" code and after testing it, proceeded to implementing the
kicker and the catcher code. Through trial and error and many tests, the
software and the hardware group adjusted the robot until the first friendly
match that took place on 01/02/2017. Several issues came up such as a faulty
Arduino and also several members of the Group becoming ill. This did not affect
our deadlines time-line though as we planned for such issues in our risk
assessment (see section 5).


\begin{figure}[H]
	\centering
	\begin{minipage}{1\textwidth}
		\centering
		\includegraphics[width=16cm, height=10cm]{FirstFriendlyMatch.png}\\
		\caption{Gant chart illustrating the group tasks until the first friendly match (Milestones are indicated with symbol "*")}
	\end{minipage}%
\end{figure}

\subsection{First Match Review and tasks until second friendly match}
The results of the first match were three draws and one loss. The problematic
areas were with the vision system and our kicker as we did not manage to score.
The grabber worked well but needs improvement. For our second friendly match, we
aim to adjust the issues discussed above. Taking into account the performance of
the robot in the first friendly match, we observed that we need to change our
strategy and improve the implementation of the catcher and the kicker. The robot
was able to move around the pitch, though it found self-recognition difficult
due to the high reflection of the plates. As the vision system that we are using
is considered a successful implementation we are not planning to change it until
we observe its reaction with the new robot plates. As for the catcher and the
kicker we decided that we are going to lower the catcher's motors power when
shooting the ball and increase the kicker's power so that the ball can move on a
straight line. We are also considering changing the batteries and adding more
powerful ones as our robot needs a pack of batteries for both the wheels and the
kicker. That will decrease the weight of the robot and it will move more
smoothly on the pitch. In our second mentor meeting we discussed these
improvements with our group mentor and we analyzed the procedure of writing the
process report. We all agree that from this point, hardware group should start
contributing with the software as there are not many hardware changes need to be
made. After adjusting the software we are planning to test our robot in both
pitches before the second friendly match. We are also planning an early-group
meeting which will include members of group 9 that we are going to cooperate
with on the third friendly match.

\begin{figure}[H]
	\centering
	\begin{minipage}{1\textwidth}
		\centering
		\includegraphics[width=16cm, height=7cm]{SecondFriendlyMatch.png}\\
		\caption{Gant chart illustrating the group tasks until the second friendly match (Milestones are indicated with symbol "*")}
	\end{minipage}%
\end{figure}


\subsection{Second Match Review and Tasks until third friendly match}
*Performance of robot in second match to be added After completing the second
friendly match we created a general approach for the tasks needing completion
until the third friendly match. This match is where we are going to cooperate
with group 9. Thus, we are planning to change our tactics before the third match
and meet with the other group so that we can organise our strategy better.
Meanwhile, we are going to have two of the weekly mentor meetings and our group
meeting. While our groups are adjusting the code we are planning to have some
practice matches with other groups to observe if the two robots are cooperating
successfully in the field. Figure 5 illustrates these tasks with their
corresponding dates.
\begin{figure}[H]
	\centering
	\begin{minipage}{1\textwidth}
		\centering
		\includegraphics[width=16cm, height=7cm]{ThirdFriendlyMatch.png}\\
		\caption{Gant chart illustrating the group tasks until the third friendly match (Milestoned indicates with a *)}
	\end{minipage}%
\end{figure}

\subsection{Tasks until fourth friendly match}
Depending on the results of the third friendly match, we are going to have a
group meeting discussing our robot's progress and a team meeting involving group
9 as well, to discuss our overall performance. Then, we will possibly have to
adjust the code to improve any weaknesses of our robot and reform each group
depending on the strategies chosen. Between the third and the fourth match we
are basically going to work on the software and hardware to make our team
cooperate in the best way. If we are satisfied with our performance, allocated
members of the group will start writing the User Guide and the Technical report.
In addition, we are going to have three more mentor meetings during this period.
These tasks can be found in Figure 6.

\begin{figure}[H]
	\centering
	\begin{minipage}{1\textwidth}
		\centering
		\includegraphics[width=16cm, height=7cm]{FourthFriendlyMatch.png}\\
		\caption{Gant chart illustrating the group tasks until the fourth friendly match}
	\end{minipage}%
\end{figure}

\subsection{Tasks until final day}
Before the final day, we will have to start preparing our presentations as well
as the user guide and the technical report. Following the fourth friendly match,
we are going to have the final adjustments in our code and discuss any possible
changes on the strategy part with group 9. Until the final day we are going to
have another three meetings with our mentor to guide us towards our presentation
and any possible changes we can make. These tasks are presented in Figure 7.
\begin{figure}[H]
	\centering
	\begin{minipage}{1\textwidth}
		\centering
		\includegraphics[width=16cm, height=7cm]{FinalDay.png}\\
		\caption{Gant chart illustrating the group tasks until final day}
	\end{minipage}%
\end{figure}

\section{Risk Assessment}
A risk assessment is crucial as it forms an integral part of a successful
project. The main objectives are to identify what issues we may encounter along
the way and how we will deal and adapt to these issues. To try and discover as
many risks we may be subject to over the course of the project, during a group
meeting we sat down and brainstormed different risks we may face and thought of
how we could deal with these. This helped in our planning of how to tackle the
project. Issues which we may come across which may backtrack our progress are
shown in Figure 8. \\include; a malfunction in the hardware such an Arduino
breaking and needing replaced, needing to redesign the robot because of a very
specific issue which the robot cannot be adapted to, a group member becoming ill
and not being able to work pushing the work they were on back, a problem with
the cameras in the pitch rooms which will backtrack testing and the pitch rooms
being to busy for testing our robot again leading to delayed development.

With a well implemented risk assessment we can avoid these issues causing a
delay in our planned timescale. To take into account of these risks we have
taken several precautions. These precautions include; providing more time than
is needed to completing tasks so that if any issues were to arise we would still
be within the timescale as we have allocated time for these issues, if a group
member were to become ill then there is another group member who is working
closely with them on the same topic who can cover some of their work, and if
there are any issues with overcrowding within a pitch room we focus on other
tasks such as report writing or hardware fixes until the pitch rooms become
available again.

\section{Conclusion}
The project is currently proceeding on schedule in the second stage of it, up
until the second friendly match. It is fully expected that the project will be
completed on time. The report contained an overview of how progress is being
made on the project, what deliverables we have promised by certain dates and
progress in relation to completed and upcoming tasks. The preceding analysis of
the project has helped identify areas of improvement such as having more
knowledge around robotics would help us design a robot in less time allowing for
other improvements and also a reducing the time needed for a variety of tasks.
Overall, all group members are putting great effort into this project and have
all gained a great deal of knowledge from it.
 
\end{document}